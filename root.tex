%%%%%%%%%%%%%%%%%%%%%%%%%%%%%%%%%%%%%%%%%%%%%%%%%%%%%%%%%%%%%%%%%%%%%%%%%%%%%%%%
%2345678901234567890123456789012345678901234567890123456789012345678901234567890
%        1         2         3         4         5         6         7         8

\documentclass[letterpaper, 10 pt, conference]{ieeeconf}  % Comment this line out if you need a4paper

%\documentclass[a4paper, 10pt, conference]{ieeeconf}      % Use this line for a4 paper

\IEEEoverridecommandlockouts                              % This command is only needed if 
                                                          % you want to use the \thanks command

\overrideIEEEmargins                                      % Needed to meet printer requirements.

%In case you encounter the following error:
%Error 1010 The PDF file may be corrupt (unable to open PDF file) OR
%Error 1000 An error occurred while parsing a contents stream. Unable to analyze the PDF file.
%This is a known problem with pdfLaTeX conversion filter. The file cannot be opened with acrobat reader
%Please use one of the alternatives below to circumvent this error by uncommenting one or the other
%\pdfobjcompresslevel=0
%\pdfminorversion=4

% See the \addtolength command later in the file to balance the column lengths
% on the last page of the document

% The following packages can be found on http:\\www.ctan.org
%\usepackage{graphics} % for pdf, bitmapped graphics files
%\usepackage{epsfig} % for postscript graphics files
%\usepackage{mathptmx} % assumes new font selection scheme installed
%\usepackage{times} % assumes new font selection scheme installed
%\usepackage{amsmath} % assumes amsmath package installed
%\usepackage{amssymb}  % assumes amsmath package installed

\title{\LARGE \bf
Advances in Robotic Learning Paper Summary: \\
Learning to Plan with Logical Automata
}

\author{Brandon Araki$^{1,*}$, Kiran Vodrahalli $^{2,*}$, Thomas Leech$^{1,3}$, Cristian-Ioan Vasile$^{1}$, Mark Donahue$^{3}$, Daniela Rus $^{1}$ \\
Presented by: Frank Liu, Evan Lam, Michael Drolet
\thanks{$^{1}$MIT CSAIL, Cambridge, MA 02139}% <-this % stops a space
\thanks{$^{2}$Columbia University, New York City, NY 10027 }%
\thanks{$^{2}$MIT Lincoln Laboratory, Lexington, MA 02421}%
\thanks{$^{*}$Authors contributed equally}%
}


\begin{document}



\maketitle
\thispagestyle{empty}
\pagestyle{empty}


%%%%%%%%%%%%%%%%%%%%%%%%%%%%%%%%%%%%%%%%%%%%%%%%%%%%%%%%%%%%%%%%%%%%%%%%%%%%%%%%
\begin{abstract}

This electronic document is a live template. The various components of your paper [title, text, heads, etc.] are already defined on the style sheet, as illustrated by the portions given in this document.

\end{abstract}


%%%%%%%%%%%%%%%%%%%%%%%%%%%%%%%%%%%%%%%%%%%%%%%%%%%%%%%%%%%%%%%%%%%%%%%%%%%%%%%%
\section{INTRODUCTION}

Imagine learning a skill like driving. In order to drive properly you not only need to learn how to drive a car, but also learn the rules of the road. To learn how to drive a car, many of us practiced driving and learning all the mechanics to make the car move. Most of us went to a driving school, where a driving instructor taught us certain driving rules in the United States. Others might watch instructional videos from experts online. One way or another, we developed a mental model of the rules of the road through imitating an expert. 
\newline
\indent Now there are two parts to this learning. The first part is learning the lower level actions in order to operate and drive a car. The second part is developing a mental model or representation of an interpretable policy, such as the rules of the road. The structure of the learned policy should be grounded in meaningful interpretations.
\newline
\indent When learning the rules of the road, a naive assumption is that all experts have taught properly and that all the instruction received is correct. If there are bad or even illegal driving habits, these will need to be corrected to ensure safe driving. In real life if a person runs a red light or makes illegal u-turns, after a certain point a police officer would come and help correct that behavior (through a ticket or more serious consequences). We are able to be corrected because the rules in our heads are manipulable, where a human operator can easily modify a learned policy to perform similar but different policies. 
\newline
\indent Applying this to robotic learning, the authors work towards teaching a robot to learn from demonstrations not just a low-level policy, but also a high level policy that is interpretable and manipulable. The authors create a Logic-based Value Network (LVIN) which utilize these two principles in learning policies. The policies that a robot learns should be interpretable, where there is a set of learned representation of rules. The behavior of the robot should be manipulable, where the rules can be changed in a predictable way which results in changed behavior. The LVIN model is a recurrent, convolutional neural network which uses value iteration over a learned Markov Decision Process (MDP). This MDP factors into two seperate parts, the first as a finite state automaton (FSA) corresponding to the low-level policy, and a bigger MDP corresponding to the rules in an environment.
\newline
\indent A big benefit to this approach to learning is that a robot won't just learn from demonstrations, but can modify the learned policy to be safe. Going back to the driving example, but this time with a robot, if a robot was learning the rules of the road and five percent of the training data included say illegal left turns which results in crashes then the robot would learn the policy which crashes five percent of the time. With the author's approach in robotic learning, such a policy can be corrected to stop the crashes. These rules can also be applied in many alternative scenarios. 
\newline
\indent In this paper, the authors main contributions are
\begin{enumerate}
  \item A Logic-based Value Network (LVIN) model which learns policies for robotic learning with an imitation learning goal. The authors show the effectiveness of the LVIN model through four different benchmark scenarios.
  \item The authors show that the model can learn the transitions from state to state, showing that it can interpret the rules.
  \item  The authors show that the learning is manipulable, thus generalizing to other tasks and fix mistakes without extra training or experts.
\end{enumerate}

\section{Related Work}

\subsection{Logic-based Approaches}

First, confirm that you have the correct template for your paper size. This template has been tailored for output on the US-letter paper size. 
It may be used for A4 paper size if the paper size setting is suitably modified.

\subsection{Multitask and Meta Learning}

The template is used to format your paper and style the text. All margins, column widths, line spaces, and text fonts are prescribed; please do not alter them. You may note peculiarities. For example, the head margin in this template measures proportionately more than is customary. This measurement and others are deliberate, using specifications that anticipate your paper as one part of the entire proceedings, and not as an independent document. Please do not revise any of the current designations

\subsection{Faulty Experts}

The template is used to format your paper and style the text. All margins, column widths, line spaces, and text fonts are prescribed; please do not alter them. You may note peculiarities. For example, the head margin in this template measures proportionately more than is customary. This measurement and others are deliberate, using specifications that anticipate your paper as one part of the entire proceedings, and not as an independent document. Please do not revise any of the current designations

\subsection{Hierarchical Learning}

The template is used to format your paper and style the text. All margins, column widths, line spaces, and text fonts are prescribed; please do not alter them. You may note peculiarities. For example, the head margin in this template measures proportionately more than is customary. This measurement and others are deliberate, using specifications that anticipate your paper as one part of the entire proceedings, and not as an independent document. Please do not revise any of the current designations

\section{Problem Statement}

Before you begin to format your paper, first write and save the content as a separate text file. Keep your text and graphic files separate until after the text has been formatted and styled. Do not use hard tabs, and limit use of hard returns to only one return at the end of a paragraph. Do not add any kind of pagination anywhere in the paper. Do not number text heads-the template will do that for you.

Finally, complete content and organizational editing before formatting. Please take note of the following items when proofreading spelling and grammar:

\subsection{Abbreviations and Acronyms} Define abbreviations and acronyms the first time they are used in the text, even after they have been defined in the abstract. Abbreviations such as IEEE, SI, MKS, CGS, sc, dc, and rms do not have to be defined. Do not use abbreviations in the title or heads unless they are unavoidable.

\subsection{Units}

\begin{itemize}

\item Use either SI (MKS) or CGS as primary units. (SI units are encouraged.) English units may be used as secondary units (in parentheses). An exception would be the use of English units as identifiers in trade, such as 3.5-inch disk drive.

\end{itemize}


\subsection{Equations}

The equations are an exception to the prescribed specifications of this template. You will need to determine whether or not your equation should be typed using either the Times New Roman or the Symbol font (please no other font). To create multileveled equations, it may be necessary to treat the equation as a graphic and insert it into the text after your paper is styled. Number equations consecutively. Equation numbers, within parentheses, are to position flush right, as in (1), using a right tab stop. To make your equations more compact, you may use the solidus ( / ), the exp function, or appropriate exponents. Italicize Roman symbols for quantities and variables, but not Greek symbols. Use a long dash rather than a hyphen for a minus sign. Punctuate equations with commas or periods when they are part of a sentence, as in

$$
\alpha + \beta = \chi \eqno{(1)}
$$

Note that the equation is centered using a center tab stop. Be sure that the symbols in your equation have been defined before or immediately following the equation. 

\subsection{Some Common Mistakes}
\begin{itemize}


\item In American English, commas, semi-/colons, periods, question and exclamation marks are located within quotation marks only when a complete thought or name is cited, such as a title or full quotation. When quotation marks are used, instead of a bold or italic typeface, to highlight a word or phrase, punctuation should appear outside of the quotation marks. A parenthetical phrase or statement at the end of a sentence is punctuated outside of the closing parenthesis (like this). (A parenthetical sentence is punctuated within the parentheses.)

\end{itemize}


\section{Experiments}

Use this sample document as your LaTeX source file to create your document. Save this file as {\bf root.tex}. You have to make sure to use the cls file that came with this distribution. If you use a different style file, you cannot expect to get required margins. Note also that when you are creating your out PDF file, the source file is only part of the equation. {\it Your \TeX\ $\rightarrow$ PDF filter determines the output file size. Even if you make all the specifications to output a letter file in the source - if your filter is set to produce A4, you will only get A4 output. }

It is impossible to account for all possible situation, one would encounter using \TeX. If you are using multiple \TeX\ files you must make sure that the ``MAIN`` source file is called root.tex - this is particularly important if your conference is using PaperPlaza's built in \TeX\ to PDF conversion tool.

\subsection{Headings, etc}

Text heads organize the topics on a relational, hierarchical basis. For example, the paper title is the primary text head because all subsequent material relates and elaborates on this one topic.

\subsection{Figures and Tables}

Positioning Figures and Tables: Place figures and tables at the top and bottom of columns. Avoid placing them in the middle of columns. Large figures and tables may span across both columns. Figure captions should be below the figures; table heads should appear above the tables. Insert figures and tables after they are cited in the text.

\begin{table}[h]
\caption{An Example of a Table}
\label{table_example}
\begin{center}
\begin{tabular}{|c||c|}
\hline
One & Two\\
\hline
Three & Four\\
\hline
\end{tabular}
\end{center}
\end{table}


   \begin{figure}[thpb]
      \centering
      \framebox{\parbox{3in}{We suggest that you use a text box to insert a graphic (which is ideally a 300 dpi TIFF or EPS file, with all fonts embedded) because, in an document, this method is somewhat more stable than directly inserting a picture.
}}
      %\includegraphics[scale=1.0]{figurefile}
      \caption{Inductance of oscillation winding on amorphous
       magnetic core versus DC bias magnetic field}
      \label{figurelabel}
   \end{figure}
   

Figure Labels: Use 8 point Times New Roman for Figure labels.

\section{Discussion and Analysis}
The authors mentioned was that the LVIN model works for finite grid world environments. These finite grid world environments are limited to a 2D grid. These applications might be incredibly useful integrated into a factory workplace with robots in highly controlled environments.
\newline
\indent However, the real world is three dimensional and highly unpredictable. These finite grid world environment limits LVIN's real world use case. While the authors have a real world experiment with the Jaco arm to show the LVIN model's effectiveness, the real world experiment exists in a highly controlled environment which is not representative of real world situations.
\newline
\indent A logical next step could be extending the 2D grid into a 3D grid world. We would imagine that the larger MDP for the rules of the world would add another dimension and the smaller FSA would add more states. Overall the new network model would be more complex. While the complexity of the network would increase, we do not believe the complexity change would not be too big of a problem in training on modern computers which train significantly large machine learning models.
\newline
\indent It also seems the manipulability of the learning is limited. The authors modify the state after detecting an error and to correct bad behavior. It would be quite interesting to see if the model can automatically detect errors and self correct. We imagine there can be neural network infrastructure which can be added for error detection.
\newline
\indent It would be quite interesting to explore the LVIN model with moving objects/obstacles in the 2D grid environment. How much would that effect the learned policies and how much impact would introducing such dynamic objects change the output behavior.


\section{CONCLUSIONS}

In conclusion, the authors introduce a Logic-Based Value Iteration network which can learn policies from imitation learning and demonstration. The authors tackle how to generalize a learned policy for a particular behavior to a larger set of tasks. Additionally, the authors address how to deal with incorrectly learned policies from incorrect demonstration. This network is a combination of a finite state automaton and a larger markov decision process. The LVIN network is a generalization of the Value Iteration Network, where the LVIN network learns the relevant transitions and creates a policy from the transitions of the FSA's. The key idea of the LVIN network is that a value iteration module is added to the end of a FSA and these modules get linked together. 
\newline
\indent The authors measure the LVIN network performance in four different virtual domains (Kitchen, Longterm, Pickworld, and Driving) and a real world implementation with a jaco arm. The LVIN network has 99.84 percent, 100 percent, 83.2 percent, and 99.6 percent, 89.8 percent success rates respectively. The model is shown to be accurate and effective in generalizing to new task specifications, and correcting errors. Future works can focus on expanding the model dimensionality for 3D scenarios and even self learn errors and dynamic objects in the world.

\addtolength{\textheight}{-12cm}   % This command serves to balance the column lengths
                                  % on the last page of the document manually. It shortens
                                  % the textheight of the last page by a suitable amount.
                                  % This command does not take effect until the next page
                                  % so it should come on the page before the last. Make
                                  % sure that you do not shorten the textheight too much.

%%%%%%%%%%%%%%%%%%%%%%%%%%%%%%%%%%%%%%%%%%%%%%%%%%%%%%%%%%%%%%%%%%%%%%%%%%%%%%%%



%%%%%%%%%%%%%%%%%%%%%%%%%%%%%%%%%%%%%%%%%%%%%%%%%%%%%%%%%%%%%%%%%%%%%%%%%%%%%%%%



%%%%%%%%%%%%%%%%%%%%%%%%%%%%%%%%%%%%%%%%%%%%%%%%%%%%%%%%%%%%%%%%%%%%%%%%%%%%%%%%

\begin{thebibliography}{99}

\bibitem{c1} Araki, Brandon \& Vodrahalli, Kiran \& Leech, Thomas \& Vasile, Cristian-Ioan \& Donahue, Mark \& Rus, Daniela. (2019). Learning to Plan with Logical Automata.
\bibitem{c2} W.-K. Chen, Linear Networks and Systems (Book style).	Belmont, CA: Wadsworth, 1993, pp. 123135.


\end{thebibliography}




\end{document}
